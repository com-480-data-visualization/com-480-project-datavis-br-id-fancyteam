\documentclass[12pt,a4paper]{article}
\usepackage{graphicx}
\usepackage{fancybox}
\usepackage{fancyhdr} 	    % Custom headers and footers
\usepackage{fancyref}
\usepackage{hyperref}
\usepackage{amsmath,amsfonts,amsthm} % Math packages
\newcommand{\horrule}[1]{\rule{\linewidth}{#1}} 				% Create horizontal rule command with 1 argument of height
\begin{document}

\begin{titlepage}
    \vspace*{\fill}
    \begin{center}
        \normalfont \normalsize
        \textsc{Ecole Polytechnique Fédérale de Lausanne} \\ [25pt] % Your university, school and/or department name(s)
        \textsc{Process Book} %Titre du doc
        \\ [0.4 pt]
        \horrule{0.5pt} \\[0.4cm] % Thin top horizontal rule
        \huge Data Visualization \\ % Matière
        \horrule{2pt} \\[0.5cm] % Thick bottom horizontal rule
        \includegraphics[width=6cm]{EPFLlogo}
        ~\\[0.5 cm]
        \small\textsc{CLOUX Olivier}\\[0.4cm]
        \small\textsc{COLLAUD Jonathan}\\[0.4cm]
        \small\textsc{HUEBER Hugo}\\
        ~\\
        ~\\
        ~\\
        ~\\
        \includegraphics[scale=0.3]{sceau}
    \end{center}
    \vspace*{\fill}
\end{titlepage}

\tableofcontents

\section{Debriefing}
% A project starts with a brief from a professor (or a project proposal that you develop yourself). This brief (in most cases) is ‘text/content’ based and contains a set of demands, questions or design problems. Before starting to work on a brief you need to get a good overview of what these problems are. An in depth analysis of the brief should therefore be included in the first section of your process book. This can be achieved as follows;

% 1. Read the brief carefully. What is the central theme of the brief and what are its parameters ? What are the design problems that you need to solve? Rewrite and state the design problem(s) clearly. Break it into smaller manageable parts or statements.

% 2. Develop a rich ‘picture’ of the problem by creating typographic diagram or mind map. Place the theme of the brief at the center of a sheet of paper and place (key)words/themes that are connected around it. A mind map is a graphic technique that helps unlock creative thinking.

% 3. Analysis. List as many questions as you can about the project. For example, what exactly is being asked? What materials can I use? What would be the overall look and feel of my response? Do I have a ‘point of view’ about the subject matter and what is it? Who has dealt with this ‘problem’ before?

% 4. Synthesis. Begin to answer the questions posed in your analysis - your answers will be combinations of ideas that will help you form a theory or system to begin working on the brief. Explain (in general terms) how you intend to solve the problems posed by the brief.
\section{Research}
% Arguably, the most important part of the design “process” is the research you do at the start of a project. The ‘rich picture’ and the analysis of the brief you created in step one will give you a reasonably clear idea about what material to look for when you begin your research. However, gathering research is always a surprising and organic activity - essentially you should collect and archive anything connected to the subject matter. In broad terms research generally falls into three areas:

% Content/Concept
% This covers any writing on the subject of the brief. Who has written about the subject in the past and what are their points of view. What are the established pro’s and cons? Content may include surveys, data and interviews with experts as well as printed or online texts.

% Image/Visuals
% This covers any visual material related to your theme. What does the subject matter look like and how have other ‘visual artists’ handled its visual presentation? What do you want the subject matter to look like? This includes mood boards, collected images, videos and other photographic material.

% Type
% Think about what kind of text might be required for the brief. Is it ‘linear’ like a novel, is it newspaper clippings or online text? It helps to think about the environment in which your subject matter is embedded. For example, if the general theme of the brief relates to ‘sports’ you should look at the range of typefaces employed around sporting events, do research on the fonts found on players shirts, logo’s of sport clubs and signage around stadiums etc.


% Ideally, when you present your research in a process book you should show everything: what did you read, listen to and look at? Who did you talk to? What do your notes look like? How did you decide what was most important? How did you know when you were finished with your research? And how do you organize your research—is it in folders, on a bulletin board or digital?


% Research for a project is often context specific. So if you are working on a project about packaging you might spend time in stores looking at and documenting various forms of packaging. Likewise if you are working on a project about the types of trees found in New York City you would spend time in the city parks. That being said there are also several key places that you can go to when you are researching. These include;


% The Internet
% Google and Google images will help you to do your first search. Your (key)words from the ‘rich image’ you can type in the search engine and it will lead you to a wealth of material. Be precise, try to not drift away from the subject matter, but also don’t select too little material. Absolutely make prints of articles and images online that are of your interest. Make sure you bookmark the pages that you have selected for printing.

% The Library
% Books, magazines and articles are still important sources for content and visual research. Go to the library and look up information related to your project. The library is a great source for images, since they are often larger in format and can be scanned at a higher resolution then images found on the internet. Collect anything that might be useful – written work, information diagrams, pictures. Make sure that you photocopy and write down book titles and ISBN numbers and sources.

% Personal Interviews
% Interview people who may be able to help you with your research. Keep in mind that information found on the internet is ‘controlled and edited information,’ sometimes you need more personal and in depth information. Make sure to note the date, time and name of the interviewee.

% Take Photographs
% Don’t just search for images on the web and in books, take your own pictures too - of anything that will help you in your research. For example, if you are designing a toy for a very young child you could visit a nursery and take photographs of children using toys and include this in your research section. Besides gathering factual and the documentary images, also think about a non-literal, more poetic approach to visualizing your subject matter. Remember that digital photography is a quick (and useful) medium for sketches and visual tests.


\section{Analysis}
% Once you have begun gathering your research you next need to begin to create a visual overview of the project as it evolves - organizing the material you are collecting into a coherent whole. At this stage it is important to use your design skills to create a layout for your process book. You should choose a paper format to work with (letter or tabloid, either portrait or landscape) and also choose a simple typeface to use throughout the book.

% Bear in mind that a process book, like all other projects on this course, is a design problem. Think of a way to communicate your process that unifies everything inside the book. Create a grid structure, and plan what will go on each page; print and proof the pages before you finalize them. You should also organize your book into sections - for example introduction, research, observation, discovery, brainstorming, Ideation and conclusions and include a cover with your name and the course title clearly visible.

% Organizing Text / Articles
% Read through each article and highlight the parts you find most important. Try to describe what the article is about in a few short sentences, and attach key words to it. This will help you read between the lines, and you will be able to find elements that are comparable and show similarity that at first glance might not have been visible. Use key words to group and order, try different categories to match your articles and/or the key words you have used.

% Organizing Images
% Analyze your images, put key words underneath them and make series; be sure to crop and adjust the images to highlight similarity or to create the feel of a series. Categorize for example into ‘facts’ / ‘documentary’ / ‘imaginary’ / ‘poetic’ / ‘style’ / ‘color,’ etc. Besides researching ‘visual content,’ you can also create a ‘look-and-feel’ page, a mood-board to show what styles you find fitting for your project.

% Organizing Typographic Research
% Make a selection of fonts you find fitting for the subject matter, and create a page that shows your fonts as well as arguments to defend your choices. This page may also include other visual research (see the image reference on the previous page) or be organized separately. Please note that a typeface is a family of fonts (very often by the same designer). Within a typeface there will be fonts of varying weights or other variations. E.g., light, bold, semi-bold, condensed, italic, etc. Each such variation is a different font.

% Write Conclusions
% In each section write a short text that details your findings - with conclusions. Keep in mind that you are ‘designing/visually organizing’ the research to not only get a clear overview for yourself, but also, visually track your ‘train of thoughts’ for others. It is within these ‘findings’ that you will find the answers to how to proceed in the next stage of the project.
\section{Ideation}
%At this stage you should begin questioning where your research is leading you. How will all your information and visual research turn into actionable ideas? How many ideas have you come up with? Are your ideas simply lists or more elaborate sketches? How did you decide what ideas to move forward with? Begin sketching your ideas!

% In your process book you should begin to show the editing, selection, an Try ‘connecting the dots,’ creating a story line that combines your research findings. Find several ways of approaching your project and start doing some tests, based on the conclusions you have drawn from your research. Write notes, try to describe your specific aims, and make thumbnail sketches and drawings that develop your ideas.


% Brainstorming

% In this stage of the design process you might want to check if people can follow your thought process by starting a brainstorming session with the members of your class. Showing others your process book will help to check if people understand what you are dealing with, and if your information is organized. If they can understand the goal of the project while flipping through your research, they should be informed enough to start a constructive discussion.

% The goal of brainstorming is to throw out any and all ideas related to a project, eventually leading to one or two to take further. This is certainly not the time to hold back… ideas that may seem risky, unrealistic or even downright stupid should be jotted down or sketched. Any one of them can lead to your best work.

% Write down several concepts with different starting points or angles. Be brief and clear. Try to come up with ideas that are different and are imaginative. Take small steps in developing the visualization of each concept, by slowly improving your ‘research sheets.’ For example re-shoot the kind of images you find fitting for a concept, or resize a selection from your found images in Photoshop and play with graphic representation (for example make them black/white, duo tone etc). Concepts/ideas should describe an ‘image-system’ and a ‘typography-system,’ both based on the same starting point. They should enhance each other!


% Get Inspired

% Sometimes the quickest way to get creative inspiration for a project is to see what others have done. You’re not looking to copy of course, but rather see a concept, color, shape, typeface or any other element that might spark your next great idea. Visiting museums of all types can be a great source of inspiration. Take a walk. Sometimes its best to get outside and watch the world...you never know what will spark your imagination.d ordering of your research material into more resolved ideas.
\section{Development}
% Having analyzed the brief, gathered and organized your research and brainstormed your ideas the next step is to begin developing these ideas into more resolved final solutions. In your process book you should clearly show the methodology you use for this step in the design process. How do you actually design? Do you sketch out first or build up type and images digitally? Do you have a single idea or web of references? How do you decide what to kill? How are you getting feedback and what is the criteria for paying attention to it or ignoring it? Comparing your concepts and visual sketches, check if they still relate to your research and the aims. Usually at this stage you already ‘feel’ which one of your concepts will have the most possibilities or shows the most interesting design approaches. If you have trouble judging their quality then check your aims and see if you succeeded in answering them. Make a final choice and start designing.


% Design Multiple Versions

% Now that you’ve done your research, finalized your content and approved on some sketches you can move on to the actual design phases of the design process. While you may knock out the final design in one shot, it’s usually a good idea to present at least two versions of a design. This gives you some options and allows you to combine favorite elements from each. Be sure to keep even the versions or ideas that you choose NOT to present and that you might not even like at the time in your process book, as you never know when they’ll come in handy. As you work on your final solution to the brief you might also consider the following checklist;

% Check that others understand your designs without you having to tell them what your idea or concept behind it is.

% Check if your typography and your images are consistent. Are all your type sizes legible?

% Check that your information, credits and sources are clearly indicated.

% Check if your design reflects the brief and if it fits the aims and the specifications.

% Check technical details, are your images 300dpi, test your printer, make print tests on different paper stocks.

% Revise again when running into problems. Print end results.
\section{Evaluation}
%Step six is the final appraisal of your work. It functions as a summary or conclusion at the end of your process book. Is your final solution to the brief good? It is important to show that you can recognize when something is good and can articulate why it’s good. Your process book is ultimately an argument for your end result. It says “After all of this work there is no other place that I could have ended up.” If you feel that there are things about the project that you would change or do differently than you are not done with the project and you should go back and review and revise your final solution.


\end{document}